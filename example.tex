\documentclass{report-uestc}

% 设置报告信息
\title{数据库新技术综述报告}
\author{张三}
\school{计算机科学与工程学院(网络空间安全学院)}
\major{计算机科学与技术}
\advisor{李教授}
\studentnumber{202112345678}
\date{\the\year 年\the\month 月\the\day 日}

\begin{document}

% 生成封面
\makecover

% 中文摘要
\begin{chineseabstract}
本文对数据库新技术进行了全面的综述,重点介绍了分布式数据库、NoSQL数据库、NewSQL数据库以及数据库云服务等前沿技术。通过对这些技术的分析和比较,总结了数据库技术的发展趋势和应用前景。

本文首先介绍了传统关系型数据库面临的挑战,然后详细阐述了各种新型数据库技术的特点、优势和应用场景。最后对数据库技术的未来发展方向进行了展望。

\chinesekeyword{数据库,分布式系统,NoSQL,NewSQL,云计算}
\end{chineseabstract}

% 英文摘要
\begin{englishabstract}
This paper provides a comprehensive review of new database technologies, focusing on distributed databases, NoSQL databases, NewSQL databases, and database cloud services. Through the analysis and comparison of these technologies, the development trends and application prospects of database technology are summarized.

This paper first introduces the challenges faced by traditional relational databases, then elaborates on the characteristics, advantages and application scenarios of various new database technologies. Finally, the future development direction of database technology is prospected.

\englishkeyword{Database, Distributed Systems, NoSQL, NewSQL, Cloud Computing}
\end{englishabstract}

% 目录
\reporttableofcontents

% 正文开始
\reportcontent

\chapter{绪\hspace{6pt}论}

\section{研究背景}

随着互联网和大数据时代的到来,传统的关系型数据库管理系统(RDBMS)在处理海量数据、高并发访问等方面逐渐显露出其局限性。为了应对这些挑战,近年来涌现出了许多新型数据库技术。

\section{研究意义}

了解和掌握数据库新技术对于软件开发人员和系统架构师来说至关重要。这些新技术不仅能够提高系统的性能和可扩展性,还能够降低系统的维护成本。

\section{报告结构}

本报告共分为五章:第一章为绪论;第二章介绍分布式数据库技术;第三章介绍NoSQL数据库;第四章介绍NewSQL数据库;第五章为总结与展望。

\chapter{分布式数据库技术}

\section{分布式数据库概述}

分布式数据库是指物理上分散在不同地理位置,但在逻辑上是一个整体的数据库系统。它通过网络连接多个节点,实现数据的分布式存储和处理。

\section{关键技术}

\subsection{数据分片}

数据分片(Sharding)是将大型数据集分割成较小的、更易于管理的部分的过程。常见的分片策略包括:
\begin{itemize}
\item 范围分片:基于数据范围进行分片
\item 哈希分片:使用哈希函数确定数据存储位置
\item 目录分片:维护一个目录表来记录数据位置
\end{itemize}

\subsection{一致性协议}

在分布式环境中,保证数据一致性是一个重要挑战。常用的一致性协议包括:
\begin{itemize}
\item 两阶段提交协议(2PC)
\item Paxos协议
\item Raft协议
\end{itemize}

\section{典型系统}

\subsection{Google Spanner}

Google Spanner是一个全球分布式数据库系统,它提供了强一致性保证和高可用性。

\subsection{Amazon DynamoDB}

Amazon DynamoDB是一个完全托管的NoSQL数据库服务,它提供了快速且可预测的性能。

\chapter{NoSQL数据库}

\section{NoSQL概述}

NoSQL(Not Only SQL)数据库是为了解决大规模数据存储和高并发访问而设计的非关系型数据库。

\section{NoSQL分类}

\subsection{键值存储}

键值存储是最简单的NoSQL数据库类型,如Redis和Memcached。

\subsection{文档数据库}

文档数据库存储半结构化数据,如MongoDB和CouchDB。

\subsection{列族数据库}

列族数据库按列存储数据,适合大规模数据分析,如HBase和Cassandra。

\subsection{图数据库}

图数据库用于存储和查询图结构数据,如Neo4j和OrientDB。

\chapter{NewSQL数据库}

\section{NewSQL概述}

NewSQL是一类新型关系型数据库,它结合了传统SQL数据库的ACID特性和NoSQL数据库的可扩展性。

\section{代表系统}

\subsection{VoltDB}

VoltDB是一个内存关系型数据库,专为高吞吐量和低延迟的OLTP工作负载设计。

\subsection{CockroachDB}

CockroachDB是一个分布式SQL数据库,提供强一致性和高可用性。

\chapter{总结与展望}

\section{总结}

本报告对数据库新技术进行了系统的综述,介绍了分布式数据库、NoSQL数据库和NewSQL数据库等主要技术方向。

\section{未来展望}

未来数据库技术将朝着以下方向发展:
\begin{enumerate}
\item 更强的分布式能力
\item 更好的多模型支持
\item 更智能的自动化管理
\item 更深入的云原生集成\citing{wang1999sanwei}
\end{enumerate}

% 参考文献
\nocite{*}
\reportbibliography{reference}

\end{document}
